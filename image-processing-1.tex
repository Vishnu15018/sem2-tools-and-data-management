% Options for packages loaded elsewhere
\PassOptionsToPackage{unicode}{hyperref}
\PassOptionsToPackage{hyphens}{url}
%
\documentclass[
]{article}
\usepackage{amsmath,amssymb}
\usepackage{lmodern}
\usepackage{ifxetex,ifluatex}
\ifnum 0\ifxetex 1\fi\ifluatex 1\fi=0 % if pdftex
  \usepackage[T1]{fontenc}
  \usepackage[utf8]{inputenc}
  \usepackage{textcomp} % provide euro and other symbols
\else % if luatex or xetex
  \usepackage{unicode-math}
  \defaultfontfeatures{Scale=MatchLowercase}
  \defaultfontfeatures[\rmfamily]{Ligatures=TeX,Scale=1}
\fi
% Use upquote if available, for straight quotes in verbatim environments
\IfFileExists{upquote.sty}{\usepackage{upquote}}{}
\IfFileExists{microtype.sty}{% use microtype if available
  \usepackage[]{microtype}
  \UseMicrotypeSet[protrusion]{basicmath} % disable protrusion for tt fonts
}{}
\makeatletter
\@ifundefined{KOMAClassName}{% if non-KOMA class
  \IfFileExists{parskip.sty}{%
    \usepackage{parskip}
  }{% else
    \setlength{\parindent}{0pt}
    \setlength{\parskip}{6pt plus 2pt minus 1pt}}
}{% if KOMA class
  \KOMAoptions{parskip=half}}
\makeatother
\usepackage{xcolor}
\IfFileExists{xurl.sty}{\usepackage{xurl}}{} % add URL line breaks if available
\IfFileExists{bookmark.sty}{\usepackage{bookmark}}{\usepackage{hyperref}}
\hypersetup{
  pdftitle={image-processing-1.R},
  pdfauthor={pc},
  hidelinks,
  pdfcreator={LaTeX via pandoc}}
\urlstyle{same} % disable monospaced font for URLs
\usepackage[margin=1in]{geometry}
\usepackage{color}
\usepackage{fancyvrb}
\newcommand{\VerbBar}{|}
\newcommand{\VERB}{\Verb[commandchars=\\\{\}]}
\DefineVerbatimEnvironment{Highlighting}{Verbatim}{commandchars=\\\{\}}
% Add ',fontsize=\small' for more characters per line
\usepackage{framed}
\definecolor{shadecolor}{RGB}{248,248,248}
\newenvironment{Shaded}{\begin{snugshade}}{\end{snugshade}}
\newcommand{\AlertTok}[1]{\textcolor[rgb]{0.94,0.16,0.16}{#1}}
\newcommand{\AnnotationTok}[1]{\textcolor[rgb]{0.56,0.35,0.01}{\textbf{\textit{#1}}}}
\newcommand{\AttributeTok}[1]{\textcolor[rgb]{0.77,0.63,0.00}{#1}}
\newcommand{\BaseNTok}[1]{\textcolor[rgb]{0.00,0.00,0.81}{#1}}
\newcommand{\BuiltInTok}[1]{#1}
\newcommand{\CharTok}[1]{\textcolor[rgb]{0.31,0.60,0.02}{#1}}
\newcommand{\CommentTok}[1]{\textcolor[rgb]{0.56,0.35,0.01}{\textit{#1}}}
\newcommand{\CommentVarTok}[1]{\textcolor[rgb]{0.56,0.35,0.01}{\textbf{\textit{#1}}}}
\newcommand{\ConstantTok}[1]{\textcolor[rgb]{0.00,0.00,0.00}{#1}}
\newcommand{\ControlFlowTok}[1]{\textcolor[rgb]{0.13,0.29,0.53}{\textbf{#1}}}
\newcommand{\DataTypeTok}[1]{\textcolor[rgb]{0.13,0.29,0.53}{#1}}
\newcommand{\DecValTok}[1]{\textcolor[rgb]{0.00,0.00,0.81}{#1}}
\newcommand{\DocumentationTok}[1]{\textcolor[rgb]{0.56,0.35,0.01}{\textbf{\textit{#1}}}}
\newcommand{\ErrorTok}[1]{\textcolor[rgb]{0.64,0.00,0.00}{\textbf{#1}}}
\newcommand{\ExtensionTok}[1]{#1}
\newcommand{\FloatTok}[1]{\textcolor[rgb]{0.00,0.00,0.81}{#1}}
\newcommand{\FunctionTok}[1]{\textcolor[rgb]{0.00,0.00,0.00}{#1}}
\newcommand{\ImportTok}[1]{#1}
\newcommand{\InformationTok}[1]{\textcolor[rgb]{0.56,0.35,0.01}{\textbf{\textit{#1}}}}
\newcommand{\KeywordTok}[1]{\textcolor[rgb]{0.13,0.29,0.53}{\textbf{#1}}}
\newcommand{\NormalTok}[1]{#1}
\newcommand{\OperatorTok}[1]{\textcolor[rgb]{0.81,0.36,0.00}{\textbf{#1}}}
\newcommand{\OtherTok}[1]{\textcolor[rgb]{0.56,0.35,0.01}{#1}}
\newcommand{\PreprocessorTok}[1]{\textcolor[rgb]{0.56,0.35,0.01}{\textit{#1}}}
\newcommand{\RegionMarkerTok}[1]{#1}
\newcommand{\SpecialCharTok}[1]{\textcolor[rgb]{0.00,0.00,0.00}{#1}}
\newcommand{\SpecialStringTok}[1]{\textcolor[rgb]{0.31,0.60,0.02}{#1}}
\newcommand{\StringTok}[1]{\textcolor[rgb]{0.31,0.60,0.02}{#1}}
\newcommand{\VariableTok}[1]{\textcolor[rgb]{0.00,0.00,0.00}{#1}}
\newcommand{\VerbatimStringTok}[1]{\textcolor[rgb]{0.31,0.60,0.02}{#1}}
\newcommand{\WarningTok}[1]{\textcolor[rgb]{0.56,0.35,0.01}{\textbf{\textit{#1}}}}
\usepackage{graphicx}
\makeatletter
\def\maxwidth{\ifdim\Gin@nat@width>\linewidth\linewidth\else\Gin@nat@width\fi}
\def\maxheight{\ifdim\Gin@nat@height>\textheight\textheight\else\Gin@nat@height\fi}
\makeatother
% Scale images if necessary, so that they will not overflow the page
% margins by default, and it is still possible to overwrite the defaults
% using explicit options in \includegraphics[width, height, ...]{}
\setkeys{Gin}{width=\maxwidth,height=\maxheight,keepaspectratio}
% Set default figure placement to htbp
\makeatletter
\def\fps@figure{htbp}
\makeatother
\setlength{\emergencystretch}{3em} % prevent overfull lines
\providecommand{\tightlist}{%
  \setlength{\itemsep}{0pt}\setlength{\parskip}{0pt}}
\setcounter{secnumdepth}{-\maxdimen} % remove section numbering
\ifluatex
  \usepackage{selnolig}  % disable illegal ligatures
\fi

\title{image-processing-1.R}
\author{pc}
\date{2021-05-01}

\begin{document}
\maketitle

\begin{Shaded}
\begin{Highlighting}[]
\FunctionTok{library}\NormalTok{(magick)}
\end{Highlighting}
\end{Shaded}

\begin{verbatim}
## Linking to ImageMagick 6.9.12.3
## Enabled features: cairo, freetype, fftw, ghostscript, heic, lcms, pango, raw, rsvg, webp
## Disabled features: fontconfig, x11
\end{verbatim}

\begin{Shaded}
\begin{Highlighting}[]
\NormalTok{frink}\OtherTok{\textless{}{-}}\FunctionTok{image\_read}\NormalTok{(}\StringTok{"C:/Users/pc/OneDrive/Pictures/Camera Roll/WIN\_20210310\_11\_22\_47\_Pro.jpg"}\NormalTok{)}
\FunctionTok{print}\NormalTok{(frink)}
\end{Highlighting}
\end{Shaded}

\begin{verbatim}
##   format width height colorspace matte filesize density
## 1   JPEG   960    540       sRGB FALSE   149655   96x96
\end{verbatim}

\includegraphics[width=13.33in]{image-processing-1_files/figure-latex/unnamed-chunk-1-1}

\begin{Shaded}
\begin{Highlighting}[]
\FunctionTok{image\_rotate}\NormalTok{(frink,}\SpecialCharTok{+}\DecValTok{60}\NormalTok{)}
\end{Highlighting}
\end{Shaded}

\includegraphics[width=13.19in]{image-processing-1_files/figure-latex/unnamed-chunk-1-2}

\begin{Shaded}
\begin{Highlighting}[]
\FunctionTok{print}\NormalTok{(frink)}
\end{Highlighting}
\end{Shaded}

\begin{verbatim}
##   format width height colorspace matte filesize density
## 1   JPEG   960    540       sRGB FALSE   149655   96x96
\end{verbatim}

\includegraphics[width=13.33in]{image-processing-1_files/figure-latex/unnamed-chunk-1-3}

\begin{Shaded}
\begin{Highlighting}[]
\FunctionTok{image\_crop}\NormalTok{(frink,}\StringTok{"270X300+350"}\NormalTok{)}
\end{Highlighting}
\end{Shaded}

\includegraphics[width=3.75in]{image-processing-1_files/figure-latex/unnamed-chunk-1-4}

\begin{Shaded}
\begin{Highlighting}[]
\NormalTok{img1}\OtherTok{=}\FunctionTok{image\_scale}\NormalTok{(frink,}\StringTok{"400"}\NormalTok{)}
\FunctionTok{print}\NormalTok{(img1)}
\end{Highlighting}
\end{Shaded}

\begin{verbatim}
##   format width height colorspace matte filesize density
## 1   JPEG   400    225       sRGB FALSE        0   96x96
\end{verbatim}

\includegraphics[width=5.56in]{image-processing-1_files/figure-latex/unnamed-chunk-1-5}

\begin{Shaded}
\begin{Highlighting}[]
\FunctionTok{image\_border}\NormalTok{(img1,}\AttributeTok{color =} \StringTok{"red"}\NormalTok{,}\AttributeTok{geometry =} \StringTok{"10x10"}\NormalTok{)}
\end{Highlighting}
\end{Shaded}

\includegraphics[width=5.83in]{image-processing-1_files/figure-latex/unnamed-chunk-1-6}

\begin{Shaded}
\begin{Highlighting}[]
\FunctionTok{image\_background}\NormalTok{(img1, }\StringTok{"blue"}\NormalTok{, }\AttributeTok{flatten =} \ConstantTok{TRUE}\NormalTok{)}
\end{Highlighting}
\end{Shaded}

\includegraphics[width=5.56in]{image-processing-1_files/figure-latex/unnamed-chunk-1-7}

\begin{Shaded}
\begin{Highlighting}[]
\FunctionTok{image\_modulate}\NormalTok{(img1, }\AttributeTok{brightness =} \DecValTok{95}\NormalTok{, }\AttributeTok{saturation =} \DecValTok{60}\NormalTok{, }\AttributeTok{hue =} \DecValTok{100}\NormalTok{)}
\end{Highlighting}
\end{Shaded}

\includegraphics[width=5.56in]{image-processing-1_files/figure-latex/unnamed-chunk-1-8}

\end{document}
